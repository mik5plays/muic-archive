\documentclass[a4paper, 11pt]{article}

\usepackage{amsmath}
\usepackage{amssymb}
\usepackage{fancyhdr}
\usepackage{listings}
\usepackage{xcolor}

\usepackage[margin=1in]{geometry}

\newcommand{\question}[2] {\vspace{.25in} \hrule\vspace{0.5em}
\noindent{\bf #1: #2} \vspace{0.5em}
\hrule \vspace{.10in}}
\renewcommand{\part}[1] {\vspace{.10in} {\bf (#1)}}

\newcommand{\myname}{Theeradon Sarawek}
\newcommand{\myemail}{theeradon.sar@student.mahidol.edu}
\newcommand{\myhwnum}{4}

\setlength{\parindent}{0pt}
\setlength{\parskip}{5pt plus 1pt}
 
\pagestyle{fancyplain}
\lhead{\fancyplain{}{\textbf{HW\myhwnum}}}      % Note the different brackets!
\rhead{\fancyplain{}{\myname\\ \myemail}}
\chead{\fancyplain{}{ICCS208}}

\begin{document}

\medskip                        % Skip a "medium" amount of space
                                % (latex determines what medium is)
                                % Also try: \bigskip, \littleskip

\thispagestyle{plain}
\begin{center}                  % Center the following lines
{\Large ICCS208: Assignment \myhwnum} \\
\myname \\
\myemail \\
26 October 2024 \\
\end{center}

\lstset{
    language=Java,                 
    basicstyle=\texttt\footnotesize, 
    keywordstyle=\color{magenta}\bfseries, 
    commentstyle=\color{gray},     
    stringstyle=\color{green},                  
    tabsize=4,                     
    showspaces=false,              
    showstringspaces=false,        
    breaklines=true,               
    breakatwhitespace=true,        
}

\question{1}{Task 1: Missing Tile}
Any $2^n$-by-$2^n$ grid with one painted cell can be tiled using L-shaped triominoes such that the entire grid is covered by triominoes but no triominoes overlap with each other nor the painted cell.

\part{1}{ Subtask 1}

To prove this by induction, we assume the predicate will be that some $2^i$ by $2^i$ grid with one painted cell anywhere (since we want to be more general/open) is tileable.

BC: A $2^1$ by $2^1$ grid is tileable as with one painted cell, we know that it can fit one L-shaped trimino into it. There are four possible configurations, noted by the missing tile in each corner.

IS: Assume a P(k) = $2^k$ by $2^k$ grid is tileable with one painted cell. In theory, this should mean a P(k+1) grid, or $2^{k+1}$ by $2^{k+1}$ grid is tileable. To prove this, we can write the following:

\begin{align}
2^{k+1} \cdot 2^{k+1} = 2(2^k) \cdot 2(2^k)
\\ = (2 \cdot 2) \cdot (2^k \cdot 2^k)
\\ = 4(2^k \cdot 2^k)
\end{align}

This implies that a grid of $2^{k+1}$ by $2^{k+1}$ is composed of four of our inductive hypothesis grids. 

We assume that three out of four of these grids will have a corner piece painted, like in our base case. This means the painted corner pieces can be filled with an L-shape. The fourth grid will have the painted tile, which can be anywhere in the fourth grid.

\question{2}{Task 4: Tail Sum Of Squares}
Consider a function $sumSqr$ and its helper function $sumHelper$:
\begin{lstlisting}
int sumHelper(int n , int a) {
	if (n==0) return a;
	else return sumHelper(n-1, a + n*n);
}
int sumSqr(int n) { return sumHelper(n, 0); }
\end{lstlisting}

We want to prove that for n $\ge$ 1, sumSqr(n),→ $1^2+2^2+3^2+...+n^2$. First, let's use mathematical induction to prove that sumHelper does what is intended. 


Consider our predicate: $\quad$ sumSqr(n) $\equiv \forall$a, sumHelper(n, a) $\to a + \sum_{i=1}^{n}i^2$

B.C: $\quad$ sumSqr(0) $\equiv$ sumHelper(0, 0) $\to 0$

I.S: Assume for $n >= 1$ that sumSqr(n - 1) $\equiv \forall$a', sumHelper(n-1, a') $\to a' + \sum_{i=1}^{n-1}i^2$

We know that $a' = a + n^2$ since it derives from sumHelper(n, a) $\to$ (sumHelper(n-1, a + $n^2$);

This means that $a' + \sum_{i=1}^{n-1}i^2 = a + n^2 + \sum_{i=1}^{n-1}i^2$

Which, if we consider $n^2$ into the sum would lead to $a + \sum_{i=1}^{n}i^2$

Therefore, we know through mathematical induction that $sumHelper$ does what it is intended.

\question{3}{Task 5: Mysterious Function}
Consider the following Python code:
\begin{lstlisting}
def foo(n):
	assert n>=1
	if n == 1:
		return (1, 2)
	else:
		p, q = foo(n-1)
		return (q + p*n*(n+1), q*n*(n+1))
\end{lstlisting}
We want to prove for $n \ge 1$ that foo(n) $\to$ (p, q) such that $ \frac{p}{q} = 1 - \frac{1}{n+1} $

B.C: foo(1) $\to$ (1,2) such that $\frac{1}{2}=1- \frac{1}{1+1}= 1- \frac{1}{2} = \frac{1}{2}$

I.S: Assume that for some $n > 1$ that foo(n-1) $\to$ (p',q') such that $ \frac{p'}{q'} = 1 - \frac{1}{n-1+1} $, or $1 - \frac{1}{n}$ 

From the function, we can see that (p, q) = $(q' + p'*n(n+1), q'*n(n+1))$

Now let's begin proving.

\begin{align}
\frac{q' + p'(n)(n+1)}{q'(n)(n+1)}
= \frac{q'}{(q')(n)(n+1)} + \frac{p'(n)(n+1)}{q'(n)(n+1)}
\\ \frac{1}{(n)(n+1) } + \frac{p'}{q'}
\\ \frac{1}{(n)} \cdot \frac{1}{(n+1) } + \frac{p'}{q'} 
\\ I.H \to \frac{1}{n} \cdot \frac{1}{n+1} + 1 - \frac{1}{n}
\\ =  1 + \frac{1}{n} \cdot \frac{1}{n+1} - \frac{1}{n}
\\ =  1 + \frac{1}{n} \cdot (\frac{1}{n+1} - 1)
\\ =  1 + \frac{1}{n} \cdot (\frac{1-(n+1)}{n+1})
\\ = 1 + \frac{1}{n} \cdot \frac{-n}{n+1}
\\ = 1 - \frac{1}{n+1}
\end{align}

Therefore, we can conclude that per mathematical induction,  $n \ge 1$ that foo(n) $\to$ (p, q) such that $ \frac{p}{q} = 1 - \frac{1}{n+1} $

\question{4}{Task 6: Midway Tower Of Hanoi}
\part{1}{ Subtask I}

We want to prove that $solve\_hanoi(n, ..., ...)$ prints exactly $2^n - 1$ lines of instructions.

B.C: $ n = 0$ would print $2^0 - 1 = 0$ lines. Likewise, $n = 1$ would print $2^1 - 1 = 1$ lines. solve\_hanoi(n-1,...,...) would do nothing since $n-1=0$ 

I.S: Assume $solve\_hanoi(n, ..., ...)$ returns $2^n - 1$ lines for some $n \ge 1$. In the function, we see $solve\_hanoi(n-1)$ called twice. Per IH we assume that one call will print out $2^{n-1} - 1$ instructions. Putting all the calls into one equation would lead to:
\begin{align}
2^{n-1} - 1 + 1 + 2^{n-1} - 1
\\ = 2^{n-1} + 2^{n-1} - 1
\\ = \frac{2^n}{2} + \frac{2^n}{2} - 1
\\ = \frac{1}{2}\cdot 2^n + \frac{1}{2}\cdot 2^n - 1
\\ = 2^n - 1
\end{align}
Therefore, per mathematical induction $solve\_hanoi(n, ..., ...)$ prints exactly $2^n - 1$ lines of instructions.
\end{document}