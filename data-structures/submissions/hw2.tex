% You should title the file with a .tex extension (hw1.tex, for example)
\documentclass[a4paper, 11pt]{article}

\usepackage{amsmath}
\usepackage{amssymb}
\usepackage{fancyhdr}

\usepackage[margin=1in]{geometry}

\newcommand{\question}[2] {\vspace{.25in} \hrule\vspace{0.5em}
\noindent{\bf #1: #2} \vspace{0.5em}
\hrule \vspace{.10in}}
\renewcommand{\part}[1] {\vspace{.10in} {\bf (#1)}}

\newcommand{\myname}{Theeradon Sarawek}
\newcommand{\myemail}{theeradon.sar@student.mahidol.edu}
\newcommand{\myhwnum}{2}

\setlength{\parindent}{0pt}
\setlength{\parskip}{5pt plus 1pt}
 
\pagestyle{fancyplain}
\lhead{\fancyplain{}{\textbf{HW\myhwnum}}}      % Note the different brackets!
\rhead{\fancyplain{}{\myname\\ \myemail}}
\chead{\fancyplain{}{ICCS208}}

\begin{document}

\medskip                        % Skip a "medium" amount of space
                                % (latex determines what medium is)
                                % Also try: \bigskip, \littleskip

\thispagestyle{plain}
\begin{center}                  % Center the following lines
{\Large ICCS208: Assignment \myhwnum} \\
\myname \\
\myemail \\
28 September 2024 \\
\end{center}

\question{1}{Task 2: Fibonacci Growth}

\part{1}{ Subtask I} 

We want to prove that:
\begin{align}
1+F_{1}+F_{2}+...+ F_{n} = F_{n+2} \qquad for \, n \ge 1
\end{align}
First, let's use the Fibonacci sequence, as proven in Discrete Math:
\begin{align}
F_{n+2} = F_{n+1} + F_{n} \qquad for \, n \ge 0
\end{align}
\\ where we know that $ F_{1} = F_{2}$ = 1.
\\ We want to assume this is true for any value of $n$, suppose $n \in \mathbb{Z}, n \ge 1$. So let's use the base case F(1) for example, 
\begin{align}
1 + F_{1} = 1 + 1 = 2
\\ F_{3} = F_{2} + F_{1}
\\ = 1 + 1
\\ = 2
\end{align}
So we now know that the base case is true. Let's assume that the induction hypothesis, using $k$, looks something like:
\begin{align}
1+F_{1}+F_{2}+...+ F_{k} = F_{k+2}
\end{align}
Now for the induction step. Say F($k+1$) is the following:
\begin{align}
1+F_{1}+F_{2}+...+ F_{k} + F_{k+1} = F_{(k+1)+2}
\end{align}
This must mean, using the induction hypothesis, that
\begin{align}
F_{k+2} + F_{k+1} = F_{k+3}
\end{align}
And as per the rule proven in Discrete Math, this matches the format exactly. In this case, we just assume that $n=k+1$. The rule can be applied as we know that k MUST be at least 1.


\part{2}{ Subtask II} 

We want to prove that:
\begin{align}
\frac{1}{F_{n}} (1+ \sum_{k=1}^{n} F_{k}) \le 3
\end{align}
Using what we already proven in Subtask I, we know that:
\begin{align}
(1+ \sum_{k=1}^{n} F_{k}) = F_{n+2}
\end{align}
This must therefore imply that:
\begin{align}
\frac{F_{n+2}}{F_{n}} \le 3
\end{align}
Let's do some plugging. To streamline things, let's list the first few fibonacci numbers:
\begin{align}
F_{1},F_{2}=1, F_{3}=2, F_{4}=3, F_{5}=5, F_{6}=8, F_{7}=13
\end{align}

Let's try for the first few values of $n$:
\begin{align}
 n=1 \to \frac{F_{1+2}}{F_{1}} = \frac{2}{1} = 2
\\ n=2 \to \frac{F_{2+2}}{F_{2}} = \frac{3}{1} = 3
\\ n=3 \to \frac{F_{3+2}}{F_{3}} = \frac{5}{2} = 2.5
\\ n=4 \to \frac{F_{4+2}}{F_{4}} = \frac{8}{3} \approx 2.67
\\ n=5 \to \frac{F_{5+2}}{F_{5}} = \frac{13}{5} = 2.6
\end{align}
We can see that the values fluctuate around 2 to 3, never exceeding the upper boundary of 3. We can assume from this that the statement has been proven. Slightly unrelated, but if we were to form a limit, we would see that it would converge to the golden ratio, or $\approx 1.618$

\part{3}{ Subtask III} 

Assuming that $n = F_{r}+1 \: \exists r \ge 2$, where $F$ refers to the Fibonacci sequence, we want to find the total number of copy steps when adding $n$ items to an array suppose we use the Fibonacci growth scheme.

We know that per the Fibonacci growth scheme, an array's size will only grow to the next Fibonacci number. This means that when an array reaches size $F_{k}$, it will grow to $F_{k+1}$ where $k$ is just an arbitrary value. At the same time, we're copying $F_{k}$ items into this new array.

From the complementary DSA textbook, their "double-the-size-when-it's-full" method accounts for the time it takes per operation, denoted by a constant $c$. For this, I will assume we won't need that (since we don't need to find the time), but the logic I'll use to solve this question will be similar to theirs.

So we know that the array expands only when it hits a size that is a Fibonacci number, and until it reaches $n$. The total copy operations can be achieved by just summing how many copies we make when we expand:
\begin{align}
F_{1}+F_{2}+F_{3}+...+F_{r} + 1
\end{align}
Remember what we proved in Subtask I. This number can just be simplified to $F_{r+2}$

That should be all we have to do.
\end{document}



 
Assume that $s$ = size of array, $c$ = number of items copied, and $p$ is our current item number. We stop when $p=n$. We know we only deal with values of $n$ that are relative to Fibonacci numbers, but let's just visualize a linear progression of numbers anyway:
\begin{align}
\\ p = 0 \to s = 1 \therefore c = 0
\\ p = 1 \to s = 2 \therefore c = 1
\\  p = 2 \to s = 3 \therefore c = 2
\\  p = 3 \to s = 5 \therefore c = 3
\\  p = 4 \to s = 5 \therefore c = 0
\\  p = 5 \to s = 8 \therefore c = 5
\end{align}
Let's focus on only when $p$ can be $n$:
\begin{itemize}
	\item When $p=2$, the array gets expanded to 3. The number of items copied is 2.
	\item When $p=3$, the array gets expanded to 5. The number of items copied is 3.
	\item When $p=4$, the array gets expanded to 5. The number of items copied is 0.
\end{itemize}