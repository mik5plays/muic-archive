\documentclass[a4paper, 11pt]{article}

\usepackage{amsmath}
\usepackage{amssymb}
\usepackage{fancyhdr}
\usepackage{listings}
\usepackage{xcolor}

\usepackage[margin=1in]{geometry}

\newcommand{\question}[2] {\vspace{.25in} \hrule\vspace{0.5em}
\noindent{\bf #1: #2} \vspace{0.5em}
\hrule \vspace{.10in}}
\renewcommand{\part}[1] {\vspace{.10in} {\bf (#1)}}

\newcommand{\myname}{Theeradon Sarawek}
\newcommand{\myemail}{theeradon.sar@student.mahidol.edu}
\newcommand{\myhwnum}{3}

\setlength{\parindent}{0pt}
\setlength{\parskip}{5pt plus 1pt}
 
\pagestyle{fancyplain}
\lhead{\fancyplain{}{\textbf{HW\myhwnum}}}      % Note the different brackets!
\rhead{\fancyplain{}{\myname\\ \myemail}}
\chead{\fancyplain{}{ICCS208}}

\begin{document}

\medskip                        % Skip a "medium" amount of space
                                % (latex determines what medium is)
                                % Also try: \bigskip, \littleskip

\thispagestyle{plain}
\begin{center}                  % Center the following lines
{\Large ICCS208: Assignment \myhwnum} \\
\myname \\
\myemail \\
14 October 2024 \\
\end{center}

\lstset{
    language=Java,                 
    basicstyle=\texttt\footnotesize, 
    keywordstyle=\color{magenta}\bfseries, 
    commentstyle=\color{gray},     
    stringstyle=\color{green},                  
    tabsize=4,                     
    showspaces=false,              
    showstringspaces=false,        
    breaklines=true,               
    breakatwhitespace=true,        
}

\question{1}{Task 3: Loops \& Numerical Computation}
Consider the following code, now with filled-in Hoare logic justifications:
\begin{lstlisting}
// precondition: x >= 0 && y >= 0;
int mult(int x, int y) {
	int k = x, n = y, res = 0;
	while (k != 0) { // @loop_invariant x * y == k * n + res;
		// { k = k0, n = n0, res = 0 for some k0, n0 }
		if (k%2 == 1) { 
			// { k%2 == 1 } and {  x * y == k0 * n0 + res }
			res = res + n;
			// {  x * y == k0 * n0 + res + n0 }
		}
		k /= 2;
		// {  x * y == (k0/2) * n0 + res } 
		n *= 2;	
		// {  x * y == k0 * (n0*2) + res }
	}
	return res;
	// { res == x * y }
}
// post-condition: returns x * y 
\end{lstlisting}

\question{2}{Task 4: Looping an Array}
Consider the following code, now with filled-in Hoare logic justifications as well as a precondition + postcondition:
\begin{lstlisting}
// precondition: A.length >= 0, x is an integer
int find(int[] A, int x) {
	int n = A.length, i = n - 1; // { n = A.length >= 0 => i = n - 1 >= 0 }
	while (i >= 0) { // @loop_invariant {i >= -1} & {n = A.length} 
		if (A[i] == x)
			// { A[i] == x }
			return i;
			// {0 <= i <= n - 1} postcondition #1
		i -= 1;
		// {i = i - 1 => i >= -1}
	}
	// { i == -1 } postcondition #2
	return -1;
}
// postcondition: return index i where {0 <= i <= n - 1} of first occurence of x counting backwards, otherwise return -1 for no occurences.
\end{lstlisting}
\end{document}