% You should title the file with a .tex extension (hw1.tex, for example)
\documentclass[a4paper, 11pt]{article}

\usepackage{amsmath}
\usepackage{amssymb}
\usepackage{fancyhdr}

\usepackage[margin=1in]{geometry}

\newcommand{\question}[2] {\vspace{.25in} \hrule\vspace{0.5em}
\noindent{\bf #1: #2} \vspace{0.5em}
\hrule \vspace{.10in}}
\renewcommand{\part}[1] {\vspace{.10in} {\bf (#1)}}

\newcommand{\myname}{Theeradon Sarawek}
\newcommand{\myemail}{theeradon.sar@student.mahidol.edu}
\newcommand{\myhwnum}{1}

\setlength{\parindent}{0pt}
\setlength{\parskip}{5pt plus 1pt}
 
\pagestyle{fancyplain}
\lhead{\fancyplain{}{\textbf{HW\myhwnum}}}      % Note the different brackets!
\rhead{\fancyplain{}{\myname\\ \myemail}}
\chead{\fancyplain{}{ICCS208}}

\begin{document}

\medskip                        % Skip a "medium" amount of space
                                % (latex determines what medium is)
                                % Also try: \bigskip, \littleskip

\thispagestyle{plain}
\begin{center}                  % Center the following lines
{\Large ICCS208: Assignment \myhwnum} \\
\myname \\
\myemail \\
20 September 2024 \\
\end{center}

\question{1}{Task 2: Oh My Magical Math (OHMM...)}

\part{1}{ Subtask II}

Consider the first $n$ even numbers (including zero):
\begin{align}
0,2,4,6,8,10,...
\end{align}
Consider that the common difference is 2, and comparing it with the sequence $2n$, which is \{2,4,6,8,10\},
We can see that the general pattern is $2n-2$ or $2(n-1)$ for the $n$th term of this sequence.
The sum of said sequence can be written like the following:
\begin{align}
S=0+2+4+6+8+...+2(n-1)
\end{align}
Writing the same sequence backwards looks like:
\begin{align}
S=2(n-1)+2(n-2)+2(n-3)+...+2+0
\end{align}
Adding these two variations of the sums together gives us:
\begin{align}
2S=(0+2(n-1))+(2+2(n-2)+(4+2(n-3))+...+(2(n-2)+2)+(2(n-1)+0)
\end{align}
This simplifies to:
\begin{align}
2S=(2n-2)+(2n-2)+(2n-2)+...+(2n-2)+(2n-2)
\end{align}
Considering that there will be $n$ occurences of $2n-2$, so:
\begin{align}
2S=n(2n-2)
\\ \therefore S=\frac{n(2n-2)}{2}
\\ = \frac{(n)(2)(n-1)}{2}
\\ = n(n-1)
\end{align}
Therefore, the sum of the first $n$ even numbers, starting from 0, is $n(n-1)$.
\\ To prove that it works, say $n=5$
The first 5 even numbers would then be 0,2,4,6,8 with the sum of that being 0+2+4+6+8=20
Putting $n=5$ in said formula would be $5(5-1)=5(4)=20$. 

\part{2}{ Subtask III}

Suppose we want to use the formula $n(n-1)$ instead of our Java function, the function would always
take the same amount of steps regardless of what $n$ is. This gives us a time complexity of $O(1)$.

\end{document}

