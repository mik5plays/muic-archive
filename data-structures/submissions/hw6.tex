% You should title the file with a .tex extension (hw1.tex, for example)
\documentclass[a4paper, 11pt]{article}

\usepackage{amsmath}
\usepackage{amssymb}
\usepackage{fancyhdr}

\usepackage[margin=1in]{geometry}

\newcommand{\question}[2] {\vspace{.25in} \hrule\vspace{0.5em}
\noindent{\bf #1: #2} \vspace{0.5em}
\hrule \vspace{.10in}}
\renewcommand{\part}[1] {\vspace{.10in} {\bf (#1)}}

\newcommand{\myname}{Theeradon Sarawek}
\newcommand{\myemail}{theeradon.sar@student.mahidol.edu}
\newcommand{\myhwnum}{6}

\setlength{\parindent}{0pt}
\setlength{\parskip}{5pt plus 1pt}
 
\pagestyle{fancyplain}
\lhead{\fancyplain{}{\textbf{HW\myhwnum}}}      % Note the different brackets!
\rhead{\fancyplain{}{\myname\\ \myemail}}
\chead{\fancyplain{}{ICCS208}}

\begin{document}

\medskip                        % Skip a "medium" amount of space
                                % (latex determines what medium is)
                                % Also try: \bigskip, \littleskip

\thispagestyle{plain}
\begin{center}                  % Center the following lines
{\Large ICCS208: Assignment \myhwnum} \\
\myname \\
\myemail \\
21 November 2024 \\
\end{center}

\question{1}{Task 4: Quick Sort Recurrence}

\part{i}
This is the recurrence relation for quicksort:
\begin{align}
f(n) = n + 1 + \frac{2}{n}(f(n-1) + f(n-2) + ... + f(1)) \qquad where f(0) = 0
\end{align}
After a few operations, which I won't copy out exactly,
\begin{align}
n \cdot f (n) = 2n +(n+1)f (n-1)
\end{align}
I understand this derivation!

\part{ii}

Suppose $g(n) = \frac{f(n)}{n+1}$, writing out what we have (per hint)
\begin{align}
\frac{n \cdot f(n)}{n(n+1)} \\
= \frac{2n +(n+1)f (n-1)}{n(n+1)} \\
= \frac{2n}{n(n+1)} + \frac{(n+1)f(n-1)}{n(n+1)} \\
= \frac{2}{n+1} + \frac{f(n-1)}{n}
\end{align}

Notice how RHS is essentially $g(n-1)$. This means that:
\begin{align}
g(n) = \frac{2}{n+1} + g(n-1) \\
= g(n-1) + \frac{2}{n+1}
\end{align}
 
\part{iii}

Now, solving for the recurrence - assuming $g(0) = 0$
\begin{align} 
g(n) = g(n-1) + \frac{2}{n+1} \\
= g(n-2) + \frac{2}{n} + \frac{2}{n+1} \\
= g(n-3) + \frac{2}{n-1} +\frac{2}{n} + \frac{2}{n+1} \\
... \\
= \frac{2}{1+1} + \frac{2}{2+1} + \frac{2}{3+1} + ... + \frac{2}{n-1} +\frac{2}{n} + \frac{2}{n+1} \\
= 2 \cdot \left[ \frac{1}{2} + \frac{1}{3} + \frac{1}{4} + ... + \frac{1}{n-1} + \frac{1}{n} + \frac{1}{n+1}\right] \\
= 2 \cdot ( H_{n+1} - 1 )
\end{align}

\part{iv}

Now, to find the closed form:
\begin{align}
g(n) = 2 \cdot (H_{n+1} - 1) = \frac{f(n)}{n+1} \\
\therefore f(n) = (n+1) \cdot 2 \cdot (H_{n+1} - 1) \\
 = (2n+2) (H_{n+1} - 1)
\end{align}
Using the fact that $H_n \le 1+ln(n)$ - this means that $H_{n+1} - 1 \le ln(n+1)$. If we subsitute this approximation into the closed-form, we get something like
\begin{align}
(2n+2) \cdot ln(n+1) \\  
= 2n \cdot ln(n+1) + 2 \cdot ln(n+1) \\
\approx n \cdot ln(n) + ln(n)
\end{align}
If we just consider the dominant terms (so we ignore the constants), we can conclude that $f(n) = O(n \cdot ln(n))$
\end{document}

