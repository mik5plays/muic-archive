% You should title the file with a .tex extension (hw1.tex, for example)
\documentclass[a4paper, 11pt]{article}

\usepackage{amsmath}
\usepackage{amssymb}
\usepackage{fancyhdr}

\usepackage[margin=1in]{geometry}

\newcommand{\question}[2] {\vspace{.25in} \hrule\vspace{0.5em}
\noindent{\bf #1: #2} \vspace{0.5em}
\hrule \vspace{.10in}}
\renewcommand{\part}[1] {\vspace{.10in} {\bf (#1)}}

\newcommand{\myname}{Theeradon Sarawek}
\newcommand{\myemail}{theeradon.sar@student.mahidol.edu}
\newcommand{\myhwnum}{6}

\setlength{\parindent}{0pt}
\setlength{\parskip}{5pt plus 1pt}
 
\pagestyle{fancyplain}
\lhead{\fancyplain{}{\textbf{HW\myhwnum}}}      % Note the different brackets!
\rhead{\fancyplain{}{\myname\\ \myemail}}
\chead{\fancyplain{}{ICCS208}}

\begin{document}

\medskip                        % Skip a "medium" amount of space
                                % (latex determines what medium is)
                                % Also try: \bigskip, \littleskip

\thispagestyle{plain}
\begin{center}                  % Center the following lines
{\Large ICCS208: Assignment \myhwnum} \\
\myname \\
\myemail \\
21 November 2024 \\
\end{center}

\question{1}{Task 1: Mathematical Truth}
"Every binary tree on $n$ nodes where each has either zero or two children has precisely $\frac{n+1}{2}$ leaves."

\part{1}{ Base case}
\begin{align}
P(1) = \frac{1+1}{2} = 1
\end{align}
A binary tree with only one node, which therefore has no subtrees, has one leaf, which is also the root. Therefore, this base case holds true.

\part{2}{ Inductive step}

First and foremost, if a tree on $k$ nodes has zero children, then we can deduce that it will have $\frac{k+1}{2}$ leaves since we don't have to worry about children.

Now suppose we have two children. Suppose a $k$ node binary tree has $\frac{k+1}{2}$ leaves (per IH). Let's assume that this $k$-node tree is part of a larger tree, with $n$ nodes, such that the total amount of nodes in this tree will be $1 + k + (n-k-1) = n$ nodes.
\begin{itemize}
	\item Per strong induction, we know that since $k < n$ and $(n-k-1) < n$ it means that $k \to \frac{k+1}{2}$ leaves and $(n-k-1) \to \frac{n-k}{2}$ leaves.
	\item If we sum these together, we can get the total number of leaves:
\end{itemize}
\begin{align}
Total = \frac{k+1}{2} + \frac{n-k}{2} = \frac{k+1+n-k}{2} == \frac{n+1}{2}
\end{align}
This aligns with our initial statement, therefore we've proven it.
\end{document}